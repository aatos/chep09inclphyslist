\begin{appendix}
\section{Misc. notes}
We have demonstrated that implementations of INCL and ABLA models in Geant4 provide
an reasonable description of spallation reactions in the 100 MeV - 2 GeV incident energy range 
and that neutron cross sections are reproduced in most cases accurately.



Limitations of INCL model


Future improvements of INCL model include


A remodelling of INCL code in Geant4 is in progress.  


%The results of the direct measurements of the ::: flux are in good agreement 
%with the results predicted from the ::: model
%calculations made A.B, and C.D (KEK) and other collaborators.

%It shows that   it also shows that



\lstset{ %
language=bash,                % choose the language of the code
basicstyle=\footnotesize,       % the size of the fonts that are used for the code
numbers=left,                   % where to put the line-numbers
numberstyle=\footnotesize,      % the size of the fonts that are used for the line-numbers
stepnumber=2,                   % the step between two line-numbers. 
                                %If it's 1 each line will be numbered
numbersep=5pt,                  % how far the line-numbers are from the code
showspaces=false,               % show spaces adding particular underscores
showstringspaces=false,         % underline spaces within strings
showtabs=false,                 % show tabs within strings adding particular underscores
frame=single,                   % adds a frame around the code
tabsize=2,                      % sets default tabsize to 2 spaces
captionpos=t,                   % sets the caption-position to bottom
breaklines=true,                % sets automatic line breaking
breakatwhitespace=false,        % sets if automatic breaks should only happen at whitespace
escapeinside={\%*}{*)},          % if you want to add a comment within your code
%caption=Bash function to release a directory., 
label=listing:relRef
}

\end{appendix}