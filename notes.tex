\begin{appendix}
\section{Misc. notes}
We have demonstrated that implementations of INCL and ABLA models in Geant4 provide
an reasonable description of spallation reactions in the 200 MeV - 3 GeV incident energy range 
and that neutron cross sections are reproduced in most cases accurately.



Limitations of INCL model:
\begin{itemize}
\item Energy range: 200 MeV - 3 GeV
\item Projectiles: Protons, Neutrons, Pions, Deuterons, Tritons, He3, Alphas
\item Targets: from Aluminum (maybe even Carbon) up to Uranium
\item Targets from Deuteron to Boron supported but not extensively tested.
\item Hydrogen (i.e. proton) targets not supported.
\end{itemize}

Future improvements of INCL model include:
\begin{itemize}
\item Light target handling (including the use of Geant4 Fermi break-up)
\item Light ion projectiles (up to Carbon or even Oxygen)
\end{itemize}


A remodelling of INCL code in Geant4 is in progress.  

\subsection{Physics list design}

\subsubsection{Requirements}
Requirements of the spallation users.

List of requirements:
\begin{enumerate}
\item Validation tool: Spallation example application we use for validation purposes.
\item Accurate neutron production
\item Nucleus fragment production
\item ...
\end{enumerate}

\subsubsection{Implementation}
Since INCL/ABLA covers only one part of the energy range and target
materials, it must be complemented by other models. For energies higher
than 3 GeV the options seem to be QGSP, FTF, Bertini and Binary.

For energies lower than, say, 150 MeV probably using PreCompound might
be a good choice.

Hydrogen-1 (proton) targets (and maybe some other light targets as
well) have to be handled by something else than INCL/ABLA.

For documentation, {\tt Doxygen} seems to be the most promising tool.

\subsubsection{Validation}

Validation tools, strategy, datasets, and results.

%The results of the direct measurements of the ::: flux are in good agreement 
%with the results predicted from the ::: model
%calculations made A.B, and C.D (KEK) and other collaborators.

%It shows that   it also shows that



\lstset{ %
language=bash,                % choose the language of the code
basicstyle=\footnotesize,       % the size of the fonts that are used for the code
numbers=left,                   % where to put the line-numbers
numberstyle=\footnotesize,      % the size of the fonts that are used for the line-numbers
stepnumber=2,                   % the step between two line-numbers. 
                                %If it's 1 each line will be numbered
numbersep=5pt,                  % how far the line-numbers are from the code
showspaces=false,               % show spaces adding particular underscores
showstringspaces=false,         % underline spaces within strings
showtabs=false,                 % show tabs within strings adding particular underscores
frame=single,                   % adds a frame around the code
tabsize=2,                      % sets default tabsize to 2 spaces
captionpos=t,                   % sets the caption-position to bottom
breaklines=true,                % sets automatic line breaking
breakatwhitespace=false,        % sets if automatic breaks should only happen at whitespace
escapeinside={\%*}{*)},          % if you want to add a comment within your code
%caption=Bash function to release a directory., 
label=listing:relRef
}

\section{PLANS}
\begin{itemize}
\item 090330 Post meeting updates to poster and archieving.
\item 090324 Poster presentation at CHEP'09 Prague.
\end{itemize}

\section{HISTORY}
\begin{itemize}
\item 090508 All meterial from poster added.
\item 090506 Merge with PK. Adding material to paper form {\tt poster.tex}
\item 090504 Merge with PK. JPSC template added from \url{http://www.iop.org/EJ/journal/-page=extra.3/1742-6596} and
related Makefile target {\em paper} added.
\item 090317 No comments from GF. Last tuning and submitting to printing. (Ready on Thursday)
\item 090316  PK added  fixed image and comments from AB. 
\item 090313 Based on Physics Day poster prepared with PK the poster, 
             and sent reuest to comments for Alain and Gunter.
\end{itemize}
\end{appendix}
