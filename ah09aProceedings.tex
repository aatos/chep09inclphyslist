%\renewcommand{\captionfont}{\textit}
%\documentclass[twocolumn,twoside,slac,floatfix]{revtex4}
%\documentclass[twocolumn,twoside,floatfix]{revtex4}
\newif\ifCITENOTE  % define variable example
\CITENOTEtrue

\newif\ifPAPER  % define variable example
\PAPERtrue % set example=true/ralse
%\PAPERfalse  % if this active  use GNUmakefile  target slides

\ifPAPER
\documentclass[twoside,floatfix,a4wide]{revtex4}
\usepackage{multirow} % common packages

\usepackage{url}
\usepackage{listings}
%\documentclass[twoside,floatfix,a4    ]{revtex4}


\else   % SLIDEs show properly if you run m and m pdf

\documentclass[slidestop,compress,xdvips,9pt]{beamer} % packages spesific to slides
%Use mathserif-option above if you dont want sf style for equations
%Font-size 11pt is default. Also available 8pt,9pt, and 10pt 
\usetheme{Antibes}
\usecolortheme{lily}

\usepackage{graphicx}
\usepackage{hyperref}
\usepackage{hyperref}

\transglitter[direction=315]
\xdefinecolor{ahcol}{rgb}{0.2, 0.4, 0.1}
\xdefinecolor{olive}{cmyk}{0.64,0,0.95,0.4}
\colorlet{structure}{green!60!black} % for color substitution
%Predefined colors: red, green, blue, cyan, magenta, yellow, black,
%darkgray, gray, lightgray, orange, violet, purple, and brown


\hypersetup{
    a4paper, % page format
    pdftitle={My Title},                  % Title
    pdfsubject={Subject of the document}, % Subject 
    pdfauthor={Author name},              % Author
    pdfkeywords={list of keywords},       % Keywords
    plainpages=false, %
    colorlinks,       % links are colored
    urlcolor=blue,    % color of external links
    linkcolor=red,    % color of internal links
    citecolor=black,  % color of links to bibliography
    bookmarksnumbered
}

\usecolortheme[named=ahcol]{structure}
\useoutertheme{myinfolines}
\useinnertheme{rounded}
%\usecolortheme[named=yellow]{structure} % clashes with \usetheme
\setbeamercolor{alerted_text}{fg=blue}
\beamertemplateshadingbackground{blue!5}{yellow!10}
%\beamersetaveragebackground{blue!5}

\makeatother
%\oddsidemargin=0mm
%\evensidemargin=0mm  

%\usecolortheme[named=green]{structure}
%\xdefinecolor{ahcol}{0.3, 0.6, 0.1}

%\colorlet{structure}{green!30!gray}
\beamertemplatetransparentcoveredhigh
\title{Beamer slides embededd to {\tt d.tex}}

%Geant4 Lecture Course
\author{A.~Heikkinen \\ {\tt aatos.heikkinen@cern.ch}\\ \it{HIP, Particle Physics Seminar}}
%\institute[HIP]{On behalf of the Geant4 Collaboration}
%\date{\today}
\graphicspath{{.}{figures/}}
\begin{document}
\frame{\titlepage}
\section{}
\frame[shrink]{
\frametitle{Usage}

\transsplitverticalout<4>

For normal paper set PAPER variable to true 
and for slides set PAPER to false.
       \vspace{0.5cm}
\structure{ % blue color
:::::
}
\vspace{0.5cm}

$$\int_i^2 \sin s^2 dx$$
\structure{textStructure}
\alert{textAlert}



Finally
      ::::
}

\section{Section}
\frame{
\frametitle{From title}
\begin{itemize}
\pause \item Every thing
\pause \item that has
\pause \item beginning
\pause \item has end.
\end{itemize}

%\vspace{-4cm}
\begin{columns}
\begin{column}{0.65\textwidth}
\begin{tabular}{lcccc}
 Class & A & B & C & D \\\hline
 X & 1 & 2 & 3 & 4 \pause \\
 Y & 3 & 4 & 5 & 6 \pause \\
 Y & 3 & 4 & 5 & 6 \pause \\
 Y & 3 & 4 & 5 & 6 \pause \\
 Y & 3 & 4 & 5 & 6 \pause \\
 Y & 3 & 4 & 5 & 6 \pause \\
 Z&5&6&7&8
\end{tabular}

 \end{column}
 \begin{column}{0.35\textwidth}
%\includegraphics[scale=0.2]{images/hsswPoster.png}  

\end{column}
\end{columns}

Image:

\structure{(G.~Mavromanolakis and D. Ward, {\sf [arXiv/physics:0409040]})}
}


\frame{
\frametitle{From title}
\begin{itemize}
\item <+-| alert@+> Every thing
\item <+-| alert@+> that has
\item <+-| alert@+> its basis on
\end{itemize}
}

\def\hilite<#1>{
  \temporal<#1>{\color{gray}}{\color{blue}} {\color{blue!25}}}

\frame{
\frametitle{From title}

\begin{itemize}
  \hilite<3> \item Everything
  \hilite<4> \item that has
  \hilite<5> \item beginning
  \hilite<6> \item has end.
  \hilite<7>
\end{itemize}
Something 
}

\fi

\newif\ifshort  % define variable example
\shorttrue % set example=true/ralse
%\shortfalse

%\setlength{\voffset}{-1.54cm}
%\setlength{\hoffset}{-0.54cm}
%\setlength{\evensidemargin}{1.5cm}
%\setlength{\marginparsep}{0cm}
%\setlength{\textwidth}{15cm}
%\setlength{\textheight}{24.7cm}
%\setlength{\topmargin}{1cm}
%\setlength{\headheight}{0pt}
%\setlength{\headsep}{0pt}

% The path of picture directory is described in the environmental string
%   ($LAWPICSPATH)
% The name and path of the BiBTeX bibliography file is defined in
%   ($LAWBIBNAME)


\ifPAPER
%\usepackage{epsf}
\usepackage{graphicx}
\usepackage{wrapfig}

\usepackage{makeidx}
\usepackage{fancyhdr}
%\usepackage{asymptote}
\usepackage{amsmath}
\usepackage{eurosym} % euros


%\fancypagestyle{lawbody}{
%  \renewcommand{\headrulewidth}{0.5pt}
%  \fancyhead[LE,LO]{SiBT DAQ article / draft v0.13 (\input{draft.timestamp})}
%  \fancyhead[RE,RO]{Comments to lauri.wendland@cern.ch}
%  \fancyfoot[CE,CO]{\thepage}
%}

\numberwithin{equation}{section} % Eguation numbering with section id. Reguires amsmath -package.
\pagestyle{fancy}
\fancyhead{} % clear all fields

%\fancyfoot[R]{\it {Helsinki, April 21th, 2008}} % Left Odd, Right Even 
\fancyhead[L]{\bf {November 13, 2008}} % Left Odd, Right Even 
\fancyfoot[C]{A.~Heikkinen {\em et al.}: 
{\em A Geant4 physics list for nuclear physics applications based on INCL and ABLA models }}

\fancyhead[R]{\thepage}
%\fancyhead[RO,RE]{\bf Working Notes}
%\fancyfoot{} % clear all fields
%\fancyfoot[LE,LO]{\bf Working Notes} 

\newcommand{\urltilde}[1]{\texttt{#1}} % solves the tilde problem

%\renewcommand{\headrulewidth}{0pt}
%\renewcommand{\footrulewidth}{0pt}
%\renewcommand{\sfdefault}{phv}

%\setlength{\textheight}{235mm}
%\setlength{\textwidth}{170mm}
%\setlength{\topmargin}{-20mm}

% You should use BibTeX and apsrev.bst for references

\bibliographystyle{apsrev}

\begin{document}
%\title{Geant4 physics list for applications to spallation reactions in ADS

\title{A Geant4 physics list for spallation and related nuclear physics applications 
based on INCL and ABLA models
\footnote{Paper \cite{ah09aProceedings} in preparation for CHEP 2009, 21 - 27 March 2009 Prague, Czech Republic\\
\url{http://www.particle.cz/conferences/chep2009/}}}
%\title{A Geant4 physics list for spallation reactions}

\author{{\underline{A.~Heikkinen}, P.~Kaitaniemi}} 
\affiliation{Helsinki Institute of Physics, P.O. Box 64, FIN-00014 University of Helsinki, Finland}

\author{A. Boudard}
\affiliation{CEN-Saclay, CEA-IRFU/SPhN, 91 191 Gif sur Yvette, France}

\author{G.~Folger}
\affiliation{European Organization for Nuclear Research (CERN), Switzerland}

\begin{abstract}
We present a new Geant4 physics list prepared for nuclear physics applications
in the domain dominated by spallation.
We discuss new Geant4 models based on the translation of 
INCL intra-nuclear cascade and ABLA de-excitation codes in C++   
and used in the physic list.
The INCL model is well established for targets heavier than Aluminium
and projectile energies from $\sim$ 150 MeV up to 2.5 GeV $\sim$ 3 GeV. 
Validity of the Geant4 physics list is demonstrated from the perspective of accelerator driven systems
and EURISOL project, especially with the neutron double differential cross sections and residual
nuclei production.
Foreseen improvements of the physics models for the treatment of light targets (Carbon - Oxygen)
and light ion beams (up to Carbon) are discussed.
%Improvements of the physics models are foreseen for the treatment of light targets (Carbon - Oxygen)
%and light ion beams (up to Carbon) and will be discussed.
An example application utilizing the physics list is introduced.
%Outside this domain (light nuclei and energies down to 20-30 MeV)
%the model can be tried but the capabilities have yet to be established.
%such as spallation reactions in Accelerator Driven Systems.
\end{abstract}

%\maketitle must follow title, authors, abstract
\maketitle

\thispagestyle{fancy}

% Put \label in argument of \section for cross-referencing \section{\label{}}

\ifshort

[{\bf Suggested  responsibility}: 
A.H. main author, P.K. coding, A.B. use case physics and INCL, G.F. Geant4 physics list details]

\section{INTRODUCTION \label{section:intro}}
An unique feature of Geant4 is physics list concept providing transparent access 
to various physics models. 
Often an optional models are available with specific strengths and limitations, 
so physics lists concept is used to
provide optimal set of functionality for specific use case.

In this paper we are interested on issues relevant to nuclear physics applications,
neutron production and modelling of spallation reactions.  

%ADS
This paper is organized as follows:
key Geant4 models (intra-nuclear cascade INCL or Liege cascade, and ABLA evaporation-fission model) 
applied in this work are introduced in Sec.~\ref{sec:models},
Sec.~\ref{sec:list} documents the details of physics list implementation, and
Sec.~\ref{sec:example} describes an example application utilizing the new physics list
for the use case of spallation reaction study.
Physics performance is demonstrated particularly for 
neutron  cross sections in Sec.~\ref{sec:performance} and
Sec.~\ref{sec:conclusion} concludes with outline of future models developments.

\section{INCL AND ABLA MODELS IN GEANT4} \label{sec:models}

INCL and ABLA models are casted into independent FORTRAN based Monte Carlo codes 
INCL4.2 and ABLA V3 \cite{heikkinen07mProceedings}.
Recently these implementations have been translated into C++ and codes provides
as part of Geant4 toolkit \cite{heikkinen03aPaper}. 
First beta release of INCL 4.2 and ABLA v3 was in Geant4  9.1 (December 2007).



\section{PHYSICS LIST FOR SPALLATION STUDIES} \label{sec:list}

\section{EXAMPLE APPLICATION} \label{sec:example}

\section{PHYSICS PERFORMANCE} \label{sec:performance}





\section{CONCLUSION} \label{sec:conclusion}

%\begin{lstlisting}
%\end{lstlisting}


\else
\section{ASYMPTOTE}
Embedded Asymptote graphics.

\begin{figure}
\begin{center}

\begin{asy}
size(8cm,4cm,IgnoreAspect);
//size(10cm,5cm,IgnoreAspect);
import graph;
texpreamble("\def\Arg{\mathop {\rm Arg}\nolimits}");



real ampl(real x) {return 2.5/(1+x^2);}
real phas(real x) {return -atan(x)/pi;}

scale(Log,Log);
draw(graph(ampl,0.01,10));
ylimits(0.001,100);

xaxis("$\omega\tau_0$",BottomTop,LeftTicks);
yaxis("$|G(\omega\tau_0)|$",Left,RightTicks);

picture q=secondaryY(new void(picture pic) {
		       scale(pic,Log,Linear);
		       draw(pic,graph(pic,phas,0.01,10),red);
		       ylimits(pic,-1.0,1.5);
		       yaxis(pic,"$\Arg G/\pi$",Right,red,
			     LeftTicks("$% #.1f$",
				       begin=false,end=false));
		       yequals(pic,1,Dotted);
		     });
label(q,"(1,0)",Scale(q,(1,0)),red);
add(q);

\end{asy}
\caption{Caption text 1}\label{fig:asy1}
\end{center}
\end{figure}

\begin{figure}
\begin{center}
\begin{asy}
size(4cm,0);
import feynman;


// set default line width to 0.8bp
currentpen = linewidth(0.8);

// scale all other defaults of the feynman module appropriately
fmdefaults();

// define vertex and external points

real L = 50;

pair zl = (-0.75*L,0);
pair zr = (+0.75*L,0);

pair xu = zl + L*dir(+120);
pair xl = zl + L*dir(-120);

pair yu = zr + L*dir(+60);
pair yl = zr + L*dir(-60);


// draw propagators and vertices

drawFermion(xu--zl);
drawFermion(zl--xl);

drawPhoton(zl--zr);

drawFermion(yu--zr);
drawFermion(zr--yl);

drawVertex(zl);
drawVertex(zr);

// draw momentum arrows and momentum labels

drawMomArrow(xl--zl, Relative(left));
label(Label("$k'$",2RightSide), xl--zl);

label(Label("$k$",2LeftSide), xu--zl);

drawMomArrow(zl--zr, Relative(left));
label(Label("$q$",2RightSide), zl--zr);

drawMomArrow(zr--yu, Relative(right));
label(Label("$p'$",2LeftSide), zr--yu);

label(Label("$p$",2RightSide), zr--yl);

// draw particle labels

label("$e^-$", xu, left);
label("$e^+$", xl, left);

label("$\mu^+$", yu, right);
label("$\mu^-$", yl, right);

\end{asy}
\caption{Caption text 2}\label{fig:asy2}
\end{center}
\end{figure}
See Fig. \ref{fig:asy1} and \ref{fig:asy2}.

\begin{figure}
\begin{center}
\begin{asy}[8cm,4cm,IgnoreAspect]
import graph;
import stats;
//size(400,200,IgnoreAspect);
int n=100;
real[] a=new real[n];
for(int i=0; i < n; ++i) a[i]=Gaussrand();

histogram(a,min(a),max(a),n=100,normalize=true,low=0);

draw(graph(Gaussian,min(a),max(a)),red);

xaxis("$x$",BottomTop,LeftTicks);
yaxis("$dP/dx$",LeftRight,RightTicks);
\end{asy}
\caption{Caption text}\label{fig:asy3}
\end{center}
\end{figure}

\begin{figure}
\begin{center}


\begin{asy}[8cm,4cm,IgnoreAspect]
import graph;

picture pic;
real xsize=200, ysize=140;
size(pic,xsize,ysize,IgnoreAspect);

pair[] f={(5,5),(50,20),(90,90)};
pair[] df={(0,0),(5,7),(0,5)};

errorbars(pic,f,df,red);
draw(pic,graph(pic,f),"legend",
     marker(scale(0.5mm)*unitcircle,red,FillDraw(blue),Below));

xaxis(pic,"$x$",BottomTop,LeftTicks);
yaxis(pic,"$y$",LeftRight,RightTicks);
add(pic,legend(pic),point(pic,NW),20SE,UnFill);

picture pic2;
size(pic2,xsize,ysize,IgnoreAspect);

frame mark;
filldraw(mark,scale(0.8mm)*polygon(6),green,green);
draw(mark,scale(0.8mm)*cross(6),blue);

draw(pic2,graph(pic2,f),marker(mark,markuniform(5)));

xaxis(pic2,"$x$",BottomTop,LeftTicks);
yaxis(pic2,"$y$",LeftRight,RightTicks);

yequals(pic2,55.0,red+Dotted);
xequals(pic2,70.0,red+Dotted);

// Fit pic to W of origin:
add(pic.fit(),(0,0),W);

// Fit pic2 to E of (5mm,0):
// add(pic2.fit(),(5mm,0),E);
\end{asy}
\caption{Caption text 4}\label{fig:asy4}
\end{center}
\end{figure}


\begin{figure}
\begin{center}
\begin{asy}
// Plotting with Asymptote (http://asymptote.sourceforge.net)
size(200, 150, IgnoreAspect);
import graph;

file in=line(input("d.dat"));
real[][] a=transpose(dimension(in, 0, 0));

real[] x = a[0];
real[] y = a[1];
real[] z = a[2];
real[] d = a[3];
real[] dError = a[4];

draw(graph(x, y), red);

draw(graph(x, z), blue);
draw(graph(x, d), black);

xaxis("$E$", BottomTop, LeftTicks);
yaxis("$avg. num. of particles$", LeftRight, RightTicks);
\end{asy}
\caption{LaTeX-embedded Asyptote script reads data for 
this plot from {\sf  n.dat}} \label{fig:asy5}
\end{center}
\end{figure}

See Fig. \ref{fig:asy3}, \ref{fig:asy4}, and \ref{fig:asy5}.


\section{TEST}
%$\headsto$

$\equiv$
\{ a\_i \}
\textvisiblespace

$\angle$ 

$\epsilon$ $\varepsilon$ 
$\rho$ $\varrho$ 
$\phi$ $\varphi$ 
$\theta$ $\vartheta$
$\pm$ $\times$ $\ast$ $\circ$ $\bullet$ $\leq$ $\simeq$ $\parallel$  
$\sim$ $\ll$ $\cong$ $\in$ $\propto$ 
$\cdot$  $\hbar$   $\leftarrow$ $\Leftarrow$ $\rightarrow$ $\Rightarrow$ 
$\leftrightarrow$ $\mapsto$ $\ldots$
$\exists$ $\forall$ $\infty$  $\emptyset$ $\triangle$ $\neg$ $\partial$
$\sum$ $\prod$ $\oint$ $\langle$ $\rangle$ $\|$

$\acute{a}$ $\bar{a}$ $\grave{a}$ $\vec{a}$

$\overrightarrow{abc}$

$\overline{abc}$
$\underbrace{abc}$
$\frac{abc}{xyz}$
$\xrightarrow{text}$


Standrard math $a = \sin(\alpha)$, with typographical variations: 
$\mathbf{a} = \sin(\alpha)$, $\mathrm{a} = \sin(\alpha)$, {\boldmath$ a = \sin(\alpha)$}.

\footnotesize
Some text (footnote size)
\small
Some text (small)
\normalsize
Some text(normal size)

{\rm Roman}  {\sl Slanted} {\it Italics} {\sf Sans Serif}




Conclusions: In summary :::: leads to following results: i) ::: ii) ::: iii) ::::





\section{INTRODUCTION}
\subsection{Model limits \label{subsection:modelLimits}}

The basic steps of the INC model are summarized below:

\begin{enumerate}
\item If Pauli's exclusion principle allows and $E_{particle} > E_{cutoff}$ = 2~MeV, 
step (2) is performed to transport the products.
\end{enumerate}

Fig.~\ref{fig:MC}.

%\begin{figure}
%  \includegraphics[width=60mm,keepaspectratio]{n_2.eps}
%  \caption{Feynman diagramm done with Asymptore}
%  \label{fig:asy1}
%\end{figure}


For concentric spheres $i = \{1, 2, 3\}$ with radii
$$r_{i}(\alpha_{i}) = \sqrt{C_{1}^{2} (1 - \frac{1}{A}) + 6.4} \sqrt{-log( \alpha_{i})}$$

where $\alpha_{i} = \{0.01, 0.3, 0.7\}$ and $C_{1} = 3.3836 A^{1/3}$

\begin{equation}
 f(p) = c p ^2.
\label{eq:fp}
 \end{equation}

We have defined (\ref{eq:fp}) using momentum, $p$, $\ldots$

 \begin{equation}
 \int_0^{p_F} f(p) dp = n_{p} \hspace{0.2truecm}  or \hspace{0.2truecm}   n_{n}.
 \end{equation}

The double differential cross-sections for neutrons at angles of $7.5^{\circ}$, $30^{\circ}$, $60^{\circ}$, and $150^{\circ}$.



\fi


\ifshort % select text based on logical value of variable example



\else


%\section{CONCLUSION}
%\begin{acknowledgments}

%\end{acknowledgments}
%---------------------------------------------------------------
 
\fi
\section*{References}
\bibliographystyle{plain}  % Options plain, unsrt, alpha, abbrv
\bibliography{ah09aProceedings.bib} %10 p

\clearpage
\subsubsection{Notes}
\begin{appendix}
\section{Misc. notes}
We have demonstrated that implementations of INCL and ABLA models in Geant4 provide
an reasonable description of spallation reactions in the 200 MeV - 3 GeV incident energy range 
and that neutron cross sections are reproduced in most cases accurately.



Limitations of INCL model:
\begin{itemize}
\item Energy range: 200 MeV - 3 GeV
\item Projectiles: Protons, Neutrons, Pions, Deuterons, Tritons, He3, Alphas
\item Targets: from Aluminum (maybe even Carbon) up to Uranium
\item Targets from Deuteron to Boron supported but not extensively tested.
\item Hydrogen (i.e. proton) targets not supported.
\end{itemize}

Future improvements of INCL model include:
\begin{itemize}
\item Light target handling (including the use of Geant4 Fermi break-up)
\item Light ion projectiles (up to Carbon or even Oxygen)
\end{itemize}


A remodelling of INCL code in Geant4 is in progress.  

\subsection{Physics list design}

\subsubsection{Requirements}
Requirements of the spallation users.

List of requirements:
\begin{enumerate}
\item Validation tool: Spallation example application we use for validation purposes.
\item Accurate neutron production
\item Nucleus fragment production
\item ...
\end{enumerate}

\subsubsection{Implementation}
Since INCL/ABLA covers only one part of the energy range and target
materials, it must be complemented by other models. For energies higher
than 3 GeV the options seem to be QGSP, FTF, Bertini and Binary.

For energies lower than, say, 150 MeV probably using PreCompound might
be a good choice.

Hydrogen-1 (proton) targets (and maybe some other light targets as
well) have to be handled by something else than INCL/ABLA.

For documentation, {\tt Doxygen} seems to be the most promising tool.

\subsubsection{Validation}

Validation tools, strategy, datasets, and results.

%The results of the direct measurements of the ::: flux are in good agreement 
%with the results predicted from the ::: model
%calculations made A.B, and C.D (KEK) and other collaborators.

%It shows that   it also shows that



\lstset{ %
language=bash,                % choose the language of the code
basicstyle=\footnotesize,       % the size of the fonts that are used for the code
numbers=left,                   % where to put the line-numbers
numberstyle=\footnotesize,      % the size of the fonts that are used for the line-numbers
stepnumber=2,                   % the step between two line-numbers. 
                                %If it's 1 each line will be numbered
numbersep=5pt,                  % how far the line-numbers are from the code
showspaces=false,               % show spaces adding particular underscores
showstringspaces=false,         % underline spaces within strings
showtabs=false,                 % show tabs within strings adding particular underscores
frame=single,                   % adds a frame around the code
tabsize=2,                      % sets default tabsize to 2 spaces
captionpos=t,                   % sets the caption-position to bottom
breaklines=true,                % sets automatic line breaking
breakatwhitespace=false,        % sets if automatic breaks should only happen at whitespace
escapeinside={\%*}{*)},          % if you want to add a comment within your code
%caption=Bash function to release a directory., 
label=listing:relRef
}

\section{PLANS}
\begin{itemize}
\item 090330 Post meeting updates to poster and archieving.
\item 090324 Poster presentation at CHEP'09 Prague.
\end{itemize}

\section{HISTORY}
\begin{itemize}
\item 090508 All meterial from poster added.
\item 090506 Merge with PK. Adding material to paper form {\tt poster.tex}
\item 090504 Merge with PK. JPSC template added from \url{http://www.iop.org/EJ/journal/-page=extra.3/1742-6596} and
related Makefile target {\em paper} added.
\item 090317 No comments from GF. Last tuning and submitting to printing. (Ready on Thursday)
\item 090316  PK added  fixed image and comments from AB. 
\item 090313 Based on Physics Day poster prepared with PK the poster, 
             and sent reuest to comments for Alain and Gunter.
\end{itemize}
\end{appendix}


%\begin{appendix}
\section{Misc. notes}
We have demonstrated that implementations of INCL and ABLA models in Geant4 provide
an reasonable description of spallation reactions in the 200 MeV - 3 GeV incident energy range 
and that neutron cross sections are reproduced in most cases accurately.



Limitations of INCL model:
\begin{itemize}
\item Energy range: 200 MeV - 3 GeV
\item Projectiles: Protons, Neutrons, Pions, Deuterons, Tritons, He3, Alphas
\item Targets: from Aluminum (maybe even Carbon) up to Uranium
\item Targets from Deuteron to Boron supported but not extensively tested.
\item Hydrogen (i.e. proton) targets not supported.
\end{itemize}

Future improvements of INCL model include:
\begin{itemize}
\item Light target handling (including the use of Geant4 Fermi break-up)
\item Light ion projectiles (up to Carbon or even Oxygen)
\end{itemize}


A remodelling of INCL code in Geant4 is in progress.  

\subsection{Physics list design}

\subsubsection{Requirements}
Requirements of the spallation users.

List of requirements:
\begin{enumerate}
\item Validation tool: Spallation example application we use for validation purposes.
\item Accurate neutron production
\item Nucleus fragment production
\item ...
\end{enumerate}

\subsubsection{Implementation}
Since INCL/ABLA covers only one part of the energy range and target
materials, it must be complemented by other models. For energies higher
than 3 GeV the options seem to be QGSP, FTF, Bertini and Binary.

For energies lower than, say, 150 MeV probably using PreCompound might
be a good choice.

Hydrogen-1 (proton) targets (and maybe some other light targets as
well) have to be handled by something else than INCL/ABLA.

For documentation, {\tt Doxygen} seems to be the most promising tool.

\subsubsection{Validation}

Validation tools, strategy, datasets, and results.

%The results of the direct measurements of the ::: flux are in good agreement 
%with the results predicted from the ::: model
%calculations made A.B, and C.D (KEK) and other collaborators.

%It shows that   it also shows that



\lstset{ %
language=bash,                % choose the language of the code
basicstyle=\footnotesize,       % the size of the fonts that are used for the code
numbers=left,                   % where to put the line-numbers
numberstyle=\footnotesize,      % the size of the fonts that are used for the line-numbers
stepnumber=2,                   % the step between two line-numbers. 
                                %If it's 1 each line will be numbered
numbersep=5pt,                  % how far the line-numbers are from the code
showspaces=false,               % show spaces adding particular underscores
showstringspaces=false,         % underline spaces within strings
showtabs=false,                 % show tabs within strings adding particular underscores
frame=single,                   % adds a frame around the code
tabsize=2,                      % sets default tabsize to 2 spaces
captionpos=t,                   % sets the caption-position to bottom
breaklines=true,                % sets automatic line breaking
breakatwhitespace=false,        % sets if automatic breaks should only happen at whitespace
escapeinside={\%*}{*)},          % if you want to add a comment within your code
%caption=Bash function to release a directory., 
label=listing:relRef
}

\section{PLANS}
\begin{itemize}
\item 090330 Post meeting updates to poster and archieving.
\item 090324 Poster presentation at CHEP'09 Prague.
\end{itemize}

\section{HISTORY}
\begin{itemize}
\item 090508 All meterial from poster added.
\item 090506 Merge with PK. Adding material to paper form {\tt poster.tex}
\item 090504 Merge with PK. JPSC template added from \url{http://www.iop.org/EJ/journal/-page=extra.3/1742-6596} and
related Makefile target {\em paper} added.
\item 090317 No comments from GF. Last tuning and submitting to printing. (Ready on Thursday)
\item 090316  PK added  fixed image and comments from AB. 
\item 090313 Based on Physics Day poster prepared with PK the poster, 
             and sent reuest to comments for Alain and Gunter.
\end{itemize}
\end{appendix}


%\begin{thebibliography}{9}
%\bibitem{titarenko99a}
%Yu.~E.~Titarenko et al.,
%``Experimental and Computer Simulations Study of
%		  Irradiated by Intermediate Energy Protons'',
%nucl-ex/9908012, 1999
%\end{thebibliography}

%\begin{appendix}
%\begin{verbatim}
%\section{Current {\tt .bashr}}
%\begin{scriptsize}
%\include{code/bashrc} % code/code.tex created from original file with codetex target

%\section{Documenting Original Fortran Codes}

% \lstset{ %
% language=fortran,                % choose the language of the code
% basicstyle=\footnotesize,       % the size of the fonts that are used for the code
% showstringspaces=false,         % underline spaces within strings
% numbers=left,                   % where to put the line-numbers
% numberstyle=\footnotesize,      % the size of the fonts that are used for the line-numbers
% stepnumber=2,                   % the step between two line-numbers. 
%                                 % If it's 1 each line will be numbered
% numbersep=5pt,                  % how far the line-numbers are from the code
% backgroundcolor=\color{white},  % choose the background color. You must add \usepackage{color}
% showspaces=false,               % show spaces within strings adding particular underscores
% showtabs=false,                 % show tabs within strings adding particular underscores
% frame=single,                   % adds a frame around the code
% tabsize=2,                      % sets default tabsize to 2 spaces
% captionpos=b,                   % sets the caption-position to bottom
% breaklines=true,                % sets automatic line breaking
% breakatwhitespace=false,        % sets if automatic breaks should only happen at whitespace
% escapeinside={\%*}{*)}          % if you want to add a comment within your code
% }

% fortran/c++
%\lstset{language=fortran, 
%caption={\sf INCUL} source {\tt inaz.f},
%label=listing:boundary}
%\section{inaz.f}
%\input{code/inaz.f}
%\end{appendix}

\else
 


\fi % end PAPER

\end{document}
% LocalWords:  Mavromanolakis
