An unique feature of Geant4 is physics list concept providing transparent access 
to various physics models. 
Often an optional models are available with specific strengths and limitations, 
so physics lists concept is used to
provide optimal set of functionality for specific use case.

In this paper we are interested on issues relevant to nuclear physics applications,
neutron production and modelling of spallation reactions.  

%ADS

The goal of accelerator driven sub-critical reactor (Accelerator
Driven System, ADS) development is to provide safer and more effective
methods of nuclear power production. The most important benefits of
this new type of reactor include improved safety. In case of an
emergency the accelerator can be shut down which in turn shuts down
the reactor. Another important safety benefit is that the system
transmutes nuclear waste to less harmful isotopes.

The ADS is based on two main components: accelerator and spallation target.

This paper is organized as follows:
key Geant4 models (intra-nuclear cascade INCL or Liege cascade, and ABLA evaporation-fission model) 
applied in this work are introduced in Sec.~\ref{sec:models},
Sec.~\ref{sec:list} documents the details of physics list implementation, and
Sec.~\ref{sec:example} describes an example application utilizing the new physics list
for the use case of spallation reaction study.
Physics performance is demonstrated particularly for 
neutron  cross sections in Sec.~\ref{sec:performance} and
Sec.~\ref{sec:conclusion} concludes with outline of future models developments.
